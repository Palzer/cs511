\documentclass[11pt]{article}
\usepackage{enumerate}
\usepackage{amsfonts}
\usepackage{fullpage}

\begin{document}

\textbf{CSCI 511 Analysis of Algorithms II \hfill Winter 2013 - Johnson}

\begin{center}
\textbf{David Palzer \\ Homework 1}
\end{center}

\begin{enumerate}


\item Why is the master template not applicable to the following recurrence?

T($n$) = 2T($\frac{n}{b}$) + $\frac{n}{lg(n)}$\\
T(1) = K\\

\textbf{Solution:} The master template is not applicable to this recurrence because the glue falls through the holes that it has. We can see that a = 2, and b = 2. Let us look at each of the three cases of the master theorem;\\\\Case a: For this case to hold true then our glue, which is $\frac{n}{lg(n)}$, must equal $O(n^{[log_{b}a]-\epsilon})$, $\epsilon > 0$. Substituting in $a$ and $b$ this becomes\\ $O(n^{[log_{2}2]-\epsilon}) = O(n^{1-\epsilon})$\\ Assuming that $\epsilon$ is small enough then this essentially implies that the upper bound must have, at most, slightly less than linear growth. Our glue though, has linear growth. As n grows larger the glue will over take the less than linear growth and so $O(n^{[log_{b}a]-\epsilon})$ is not a proper upper bound on the glue proving that case a does not fit.\\\\Case b: Our glue must be $\Theta(n^{log_{b}a})$. Again substiuting in a and b we see that the bound becomes $\Theta(n^{1}) = \Theta(n)$. This means that\\$\frac{n}{lg(n)} \geq \Omega(n)$\\$\frac{n}{lg(n)} \leq O(n)$.\\\\Let us look at the lower bound;\\$k_{1}n \leq \frac{n}{lg(n)} = k_{1}nlg(n) \leq n$\\Dropping the constant results in $nlg(n) \leq n$ which we can see is not true as $nlg(n)$ is greater than $n$, thus case b does not hold true.\\\\Case c: Our glue must be $\Omega(n^{log_{b}a+\epsilon})$. Again substituting a and b we see that the bound becomes $\Omega(n^{1+\epsilon})$. This means that\\$\frac{n}{lg(n)} \geq n^{1+\epsilon}$\\$n \geq n^{1+\epsilon}lg(n)$. \\Such as in case b, we can see that this is clearly false and so case c does not hold true.\\\\As none of the three cases hold true, this recurrence cannot be solved with the master theorem.

\item Consider the following variation on case (b) of the master theorem.\\Let T($n) > 0$ be a nondecreasing function satisfying the following recurrence; let f($n$) $\geq$ 0.

$T(n) = aT(\frac{n}{b}) + f(n)$, if $n \geq b$\\
$T(n) \leq K$, for $1 \leq n < b$,\\
where $a \geq 1, b > 1$, and $\frac{n}{b}$ can either be either $\lceil{\frac{n}{b}}\rceil$ or $\lfloor{\frac{n}{b}}\rfloor$. Further, suppose

$f(n) = \theta(\frac{n^{log_{b}a}}{lg(n)})$.

Then the recurrence solution is $T(n) = \theta(n^{log_{b}a}*lg(lg(n)))$.\\
Prove this variation for the special case where $n=b^{k}$, for k = 0,1,2,...\\

\textbf{Solution:} Taking the expanded recurrence from the proof of the master theorem we have\\\\$T(b^k)=a^kT(1)+\sum\limits_{j=0}^{k-1}a^jf(b^{k-j})=a^kT(1)+\sum\limits_{j=1}^ka^{k-j}f(b^j)$\\\\For this case we have $f(n)=\Theta(\frac{n^{log_ba}}{lg(n)})$. That is, $N,K_1,K_2$ with $n \geq N$ implies $K_1\frac{n^{log_ba}}{lg(n)} \leq f(n) \leq K_2\frac{n^{log_ba}}{lg(n)}$. We choose $k_1$ such that $b^{k_1} > N$. Consequently, $k \geq k_1$ implies $b^k \geq b^{k_1} > N$, which entails\\\\$K_1\frac{(b^k)^{log_ba}}{lg(b^k)} \leq f(b^k) \leq K_2\frac{(b^k)^{log_ba}}{lg(b^k)}$\\$K_1\frac{(b^{log_ba})^k}{lg(b^k)} \leq f(b^k) \leq k_2\frac{(b^{log_ba})^k}{lg(b^k)}$\\$K_1\frac{a^k}{lg(b^k)} \leq f(b^k) \leq K_2\frac{a^k}{lg(b^k)}$\\$K_1\frac{a^k}{klg(b)} \leq f(b^k) \leq K_2\frac{a^k}{klg(b)}$\\$\frac{K_1}{lg(b)}\frac{a^k}{k} \leq f(b^k) \leq \frac{K_2}{lg(b)}\frac{a^k}{k}$\\\\Breaking the above expansion into components for $1 \rightarrow k_1$ and $k_1+1 \rightarrow k$\\\\$T(b^k)=K_1a^k+\sum\limits_{j=1}^{k_1}a^{k-j}f(b^j)+\sum\limits_{j=k_1+1}^{k}a^{k-j}f(b^j)$\\\\for all $k \geq k_1$. We have written $T(1) = K_1$. All three components are positive so we obtain a simple lower bound by dropping all but the last:\\\\$T(b^k) \geq \sum\limits_{j=k_1+1}^{k}a^{k-j}f(b^j) \geq \sum\limits_{j=k_1+1}^{k}a^{k-j}\frac{K_1}{lg(b)}\frac{a^j}{j} \geq \frac{K_1a^k}{lg(b)}\sum\limits_{j=k_1+1}^{k}\frac{1}{j} \geq \frac{K_1a^k}{lg(b)}lg(k)$\\\\for $k > k_1$. In terms of $n = b^k$ this bound is\\\\$T(n)\geq a^{log_bn}*lg(log_bn) = n^{log_ba}*ln(\frac{ln(n)}{ln(b)}) = n^{log_ba}*(lnln(n) - lnln(b))\\ \geq n^{log_ba}*lnln(n)$\\\\for all $n = b^k, k \geq k_1$ therefore $T(n) = \Omega(n^{log_ba}*lnln(n))$ on the contrained sequence $n = b^k$\\\\%here is the upper bound
For the matching upper bound we return to the expansion\\\\$T(b^k)=K_1a^k+\sum\limits_{j=1}^{k_1}a^{k-j}f(b^j)+\sum\limits_{j=k_1+1}^{k}a^{k-j}f(b^j)\\ = K_1a^k+a^k\sum\limits_{j=1}^{k_1}a^{-j}f(b^j)+\sum\limits_{j=k_1+1}^{k}a^{k-j}f(b^j)$\\\\for $k \geq k_1$ letting $K_3$ = max$_{1 \leq j \leq k_1}f(b^j)$ and noting $a^{-j}$ for $1 \leq j \leq k_1$, we have\\\\$T(b^k) \leq k_1a^k+a^k\sum\limits_{j=1}^{k_1}K_3+\sum\limits_{j=k_1+1}^{k}a^{k-j}f(b^j)=(K_1+k_1K_3)a^k+\sum\limits_{j=k_1+1}^{K}a^{k-j}f(b^j)$\\\\in the remaining sum we know $f(b^j) \leq \frac{K_2a^j}{jlg(b)}$\\\\$(K_1+k_1K_3)a^k+\sum\limits_{j=k_1+1}^{k}a^{k-j}\frac{K_2a^j}{jlg(b)} =(K_1+k_1K_3)a^k+\frac{K_2a^k}{lg(b)}\sum\limits_{j=k_1+1}^{k}\frac{1}{j}\\=(K_1+k_1K_3)a^k+\frac{K_2a^k}{lg(b)}2ln(k)=2(K_1+k_1K_3+\frac{K_2}{lg(b)})a^kln(k)$\\\\in terms of $n = b^k$ this result is\\\\$T(n) \leq 2(K_1+k_1K_3+\frac{K_2}{lg(b)})n^{log_ba}*lglg(n)$ for $n = b^k \geq b^{k_1}$\\Therefore $T(n) = O(n^{log_ba}*lglg(n))$ along the constrained sequence $n = b^k$.\\\\Combining both findings results in $T(n) = \Theta(N^{log_ba}*lglg(n))$

\item Extend the preceding proff to the general case in which $n$ is not necessarily a power of b.\\

\textbf{Solution:} We know that our recurrence, T(n), is a monotonically positive function. This means that it never decreases. From $n$ to $n+1$ the recurrence may or may not become larger but it will never become smaller. From this fact we can then glean that any $n$ picked between $b^j$ and $b^{j+1}$ must have an equal to or greater recurrence than $n$ at $b^j$ . Following this same logic we can see that this same $n$ must then have a recurrence smaller than or equal to the recurrence of $n$ at $b^{j+1}$. Since these are true then it must also hold that this can extend past $b^{j+1}$ to the point where $n$, $n_1 \leq n \leq n_2$ for any $n_1$, $n$, and $n_2$, must have a recurrence, $T(n_1) \leq T(n) \leq T(n_2)$, that fits this pattern no matter what $n$ is. This is only possible because of the monotonicity of the recurrence..

\end{enumerate}
\end{document}
